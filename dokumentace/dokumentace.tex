%%
% Name: IMS - Vodovody a kanalizace Vyškov
% Autors: Michal Cupak, Ales Dujicek
% Date: 2012-12-07
%%

%\documentclass[11pt,a4paper]{article}
\documentclass[12pt,a4paper]{article}
\usepackage[a4paper]{geometry}
\usepackage[czech]{babel}
%\usepackage{czech}
%\usepackage[latin2]{inputenc}
%\usepackage[IL2]{fontenc}
\usepackage[utf8]{inputenc}
%\usepackage[T1]{fontenc}
\usepackage{times}

\usepackage{graphics}
\usepackage{picture}
%\usepackage[czech,ruled]{algorithm2e}

\usepackage{amsmath}
\usepackage{amsthm}
\usepackage{amsfonts}

\usepackage{parskip}

%\newcommand{\uvoz}[1]{\quotedblbase #1\textquotedblleft}

\begin{document}

% Úvodní strana
%%
% Name: Titulní strana
% Autor: Michal Cupak
% Date: 2011-3-11
%%


\begin{titlepage}

\begin{center}
  
		{\Huge  {\scshape Vysoké učení technické v Brně}\\}
		{\huge {\scshape Fakulta informačních technologií}\\}
	
	\vspace{\stretch{0.382}}
		
	{\LARGE Modelování a simulace}
	\\
	\medskip
	{\Huge Vodovody a kanalizace Vyškov}\\
	\vspace{\stretch{0.618}}
\end{center}
{\Large \today \\\\ Michal Cupák \\ Aleš Dujíček}


\end{titlepage}


% Obsah
% \thispagestyle{empty}
% \tableofcontents
% \newpage

% Kapitola 1 - Úvod
\section{Úvod}

V této práci je řešena implementace systému hromadné obsluhy \cite{ims-sho}, která bude použita pro 
sestavení modelu \cite{ims-zakladni-pojmy} zpracování dokumentů v administrativě firmy Vodovody a kanalizace Vyškov.

Na základě modelu a simulačních experimentů bude ukázáno 
chování systému \cite{ims-chovani-systemu} toku dokumentů mezi jednotlivými zaměstnanci firmy v podmínkách běžné denní činnosti firmy.

Smyslem experimentů je demonstrovat, jak se při změně jakékoliv veličiny ovlivňující tok dokumentů ve studované firmě změní délka zpracování daného dokumentu či vy\-tí\-že\-ní ostatních zaměstnanců studované firmy.

Správnost zvolené koncepce byla ověřena přímo se zaměstnancem studované firmy.

% \textit{(pro zpracování modelu bylo nutno nastudovat ..., zpracovat, ... model je ve svém oboru zajímavý/ojedinělý v ...)}

\subsection{Kdo se na práci podílel}
Autory této práce jsou Michal Cupák a Aleš Dujíček, studenti Fakulty informačních technologií Vysokého učení technického v Brně

Fakta o společnosti nám laskavě poskytl pan Ing. Oldřich Novoměstský, vedoucí technického úseku společnosti Vodovody a kanalizace Vyškov.

Odborným konzultantem byl taktéž Ing. Novoměstský ve spolupráci s paní Annou Sý\-ko\-ro\-vou, se\-kre\-tář\-kou ředitele.

\subsection{Ověření validity modelu}
Ověřování validity modelu probíhalo při osobních setkáních s Ing. Novoměstským. Jednalo se především o vysvětlení problematiky zpracování dokumentů ve studované firmě. Dále pak o ověření funkčnosti modelu pomocí konzultace grafů vzniklých ze simulace \cite{ims-zakladni-pojmy}.

Počet a procentuální rozdělení příchozích dokumentů byl konzultován s paní Sýkorovou, která má tyto dokumenty zpočátku na starosti.


%%%%%%%%%%%%%%%%%%%%%%%%%%%%%%%%%%%%%%%%%%%%%%%%%%%%%%%%%%%%%%%%%%%%%%%%%%%%%%%%%%%%%%%%=
\newpage


% Kapitola 2 - 
\section{Rozbor tématu a použitých metod/technologií}

Administrativa společnosti Vodovody a kanalizace Vyškov zahrnuje zpracování dokumentů (stížnosti, žádosti, smlouvy a dotazy), které jim lidé posílají poštou, elektronicky emaily a datovými schránkami a doručují osobně. Tok dokumentů je znázorněn na obrázku \ref{tok_dokumentu} a popsán v následujících odstavcích.

Všechny příchozí dokumenty zpracovává sekretářka ředitele, která je musí naskenovat speciálním skenerem, vytisknout označené čárovým kódem a zaevidovat do informačního systému. Sekretářka se administrativní činnosti věnuje celou svou pracovní dobu.

Takto zpracované dokumenty prohlíží ředitel, který je dle obsahu či adresáta při\-dě\-lu\-je k vyřízení pracovníkům některému z úseků (úsek ředitele, ekonomický úsek, technický úsek). Ředitel se pro své vytížení jinými pracovními povinnostmi věnuje této administrativě průmerně hodinu denně.
Průměrně třikrát za dva týdny není ředitel k dispozici vůbec, pokud jeho nepřítomnost trvá déle než dva dny po sobě, zastoupí ho technický náměstek.

V úseku ředitele jsou dokumenty přímo přidělovány ředitelem konkrétním pra\-cov\-ní\-kům.

Dokumenty k vyřízení v ekonomickém úseku prostuduje ekonomický náměstek, který potom jejich vyřízením pověřuje své přímé podřízené.

V technickém úseku dokumenty nejprve prostuduje technický náměstek. Ten může jejich vyřízením pověřit své přímé podřízené, nebo může daný dokument předat vedoucímu technického úseku, který jej následně předá k vyřízení některému ze svých přímých podřízených.

Ve 2\% případů se dokument doručí špatnému člověku. Tudíž se musí dokument vrátit zpět k nadřízenému, který své předchozí rozhodnutí přehodnotí, případně dokument doručí řediteli či přímo náměstkovi druhého úseku. Dokument se tak déle zdrží.

Průměrně jednou až dvakrát měsíčně se objeví porucha technického zařízení, se kterým pracuje sekretářka. Ta potom po dobu opravy, která trvá průměrně 3 hodiny, nemůže dokumenty zpracovávat.

Pracovní doba zaměstnanců společnosti je 7,5 hodiny denně. Poštu přiváží pracovník pošty každý den mezi 9. a 10. hodinou, současně odveze odchozí poštu.

Denně je průmerně zaevidováno 50-60 příchozích dokumentů, které nejčastěji při\-chá\-zí klasickou poštou (20-30), dále elektronicky (20) a nejméně osobně (10).

Reálná doba doručení dokumentu k adresátovi se pohybuje okolo tří pracovních dnů. Maximální doba pak okolo pěti pracovních dnů.

Všechny informace uvedené v této kapitole byly poskytnuty zaměstnancem vedení spo\-leč\-nos\-ti, panem Ing. Oldřichem Novoměstským.

\begin{figure}[ht]
 \begin{center}
	\scalebox{0.68}{ \includegraphics{tok_dokumentu.jpg} }
	\caption{Tok dokumentů v administrativě}
	\label{tok_dokumentu}
 \end{center}
\end{figure}

\subsection{Popis použitých postupů pro vytvoření modelu}

Zaměstnance společnosti modelujeme jako zařízení. To proto, že se tak skutečně chovají, zpracovávají požadavky po jednom za sebou.

Dokumenty modelujeme jako procesy, to zase odpovídá tomu, že při jejich zpracování zabírají zařízení a přitom mají nějakou dobu obsluhy.

Příchody transakcí do systému jsou modelovány událostmi, protože nemají trvání. Po\-ru\-cha zařízení je modelována jako proces, protože trvá určitou dobu.

\subsection{Popis původu použitých metod}

Během modelování jsme využívali znalosti získané na přednáškách a demosntračních cvičeních předmětu Modelování a simulace (IMS).


%%%%%%%%%%%%%%%%%%%%%%%%%%%%%%%%%%%%%%%%%%%%%%%%%%%%%%%%%%%%%%%%%%%%%%%%%%%%%%%%%%%%%%%%=
\newpage

% Kapitola 3 - 
\section{Koncepce}

Jako jednotku času v simulaci jsme zvolili jednu hodinu.
Modelovat \cite{ims-zakladni-pojmy} budeme pouze pracovní dobu zaměstnanců. Noci, víkendy a jiné události, kvůli kterým se nepracuje, jsou vynechány.
Tedy v čase 0 začíná první pracovní den a v čase 7.5 začíná druhý bez ohledu na to, zda mezi nimi byl např. víkend.

Příchody dokumentů do systému modelujeme procesy zvlášť pro příchozí klasickou poštu, elekteronickou poštu a příchody lidí osobně.
Elektronická pošta zahrnuje emaily společně s datovými schránkami, není potřeba je rozlišovat, protože jejich zpracování probíhá totožně.
Příchody lidí jsou modelovány příchody procesů v intervalech daných exponenciálním rozdělením se středem $T/n$ \cite{ims-rozlozeni}, kde $T$ je délka pracovního dne v ho\-di\-nách a $n$ průměrný počet příchodů denně.
Obdobně je modelován příchod dokumentů elektronicky, možnost příchodu elektronické zprvávy mimo pracovní dobu (např. v noci) zanedbáváme.

Ředitel společnosti je modelován zařízením s výlučným přístupem \cite{ims-zarizeni}, které je po dobu 6.5 hodiny obsazeno procesem s vyšší prioritou obsluhy,
který představuje všechny jeho pracovní povinnosti mimo administrativu, pro kterou je po zbylou 1 hodinu uvolněn.
Ne\-za\-bý\-vá\-me se, kdy přesně se administrativě ředitel věnuje, modelujeme pouze fakt, že se této činnosti věnuje pravidelně každý den danou dobu.

S pravděpodobností 30\% ředitel uvolněn pro administrativu není, tímto modelujeme fakt, že průměrně třikrát za dva týdny není pro administrativu k dispozici (3 z 10 pracovních dnů opovídá 30\% dnů).

Poruchy zařízení sekretářky jsou modelovány procesy s příchodem s exponenciálním rozdělením se středem 100 hodin. Předpokládáme že měsíc má 20 pracovních dní, tedy 150 pracovních hodin, během kterých se objeví průměrně 1.5 poruchy.

% Konceptuální model je abstrakce reality a redukce reality na soubor relevantních faktů pro sestavení simulačního modelu.
% Pokud některé partie reality zanedbáváte nebo zjednodušujete, musí to být zdůvodněno a videálním případě musí být prokázáno, že to neovlivní validitu modelu.
% Výsledek kapitoly: konceptuální (abstraktní) model s vyznačením relevantních faktů.

%  převzetí faktů do modelu
%  zdůvodněné provedené zjednodušení faktů
%  abstraktní popis modelu/programu

% Návod: koncepci vaší práce MUSÍ pochopit libovolný technik (a často i manažer...)


%%%%%%%%%%%%%%%%%%%%%%%%%%%%%%%%%%%%%%%%%%%%%%%%%%%%%%%%%%%%%%%%%%%%%%%%%%%%%%%%%%%%%%%%=
\newpage


% Kapitola 4 - 
\section{Architektura simulačního modelu/simulátoru}

Simulační model je implementován v jazyce C++ s využitím knihovny SIMLIB/C++ \cite{simlib}.
% tohle je nejake moc strucne

% Nejméně zajímavá část. Obvykle se neuvádí.
% Rozeberte několik nejzajímavějších partií implementace
% Případná uživatelská příručka (spuštění programu, struktura výpisů, ...).
% O funkčnosti modelu musí přesvědčit kapitla 3.
% Není to referenční příručka!

% kapitola 4.1: Minimálně je nutno ukázat mapování abstraktního (koncept.) modelu do simulačního (resp. simulátoru). Např. které třídy odpovídají kterým procesům/veličinám a podobně. 
\subsection{Popis významu tříd simulačního modelu}

Příchozí dokumenty do systému v simulačním modelu jsou reprezezentovány instancemi třídy {\tt Dokument}.

Zaměstnance společnosti reprezentují zařízení {\tt reditel} - ředitel, {\tt sekretarka} - sekretářka, {\tt namestek\_tech} - technický náměstek,
{\tt namestek\_ekon} - ekonomický náměstek a {\tt vedouci\_tech} - vedoucí technického úseku.

Poruchy reprezentují události {\tt Porucha}.
Událost Email repreztuje příchod elektronického dokumentu, událost Lide společně s procesem {\tt Clovek} příchod lidí osobně.
Událost {\tt Postacka} reprezentuje příchod pracovnice pošty s poštou.

Procesy {\tt Reditel}, {\tt Namestek\_t} a {\tt Namestek\_e} modelují pracovní dobu ředitele a náměstků, kteří se nevěnují jen administrativě.

%%%%%%%%%%%%%%%%%%%%%%%%%%%%%%%%%%%%%%%%%%%%%%%%%%%%%%%%%%%%%%%%%%%%%%%%%%%%%%%%%%%%%%%%=
\newpage


% Kapitola 5 - 
\section{Podstata simulačních experimentů a jejich průběh}

Experimentováním s modelem chceme zjistit možné doby transakcí strávených v sys\-té\-mu. Pomocí těchto experimentů chceme ukázat strukturní a funkčí optimalizace systému toku dat v administrativě. Dále chceme ukázat pracovní vytíženost jednotlivých za\-měst\-nan\-ců při administrativních záležitostech. K tomu využijeme modelu a jeho pa\-ra\-met\-ri\-za\-ce, kdy například zvýšíme množství příchozí pošty, nebo poruchovost, nebo zvýšíme či snížíme pracovní výkonnost zaměstnanců.

% Experimentování musí mít předem zvolený a zdůvodněný řád, či postup
% 5.1 Postup experimentování a okolnosti studie
\subsection{Postup experimentování}
V modelu jsou již zaneseny zjištěné informace, jako například pracovní doba, množství příchozích transakcí a jejich časová rozložení, doba mezi poruchami, čas, který jednotliví pracovníci stráví nad jednou transakcí, pravděpodobnost, zda je ředitel v práci, atd. Tyto údaje jsou výrazným činitelem ovlivňujícím výsledek experimentu. Proto jsou zpočátku nastaveny na zjištěné a dále pro simulaci upravené hodnoty. Ty se budou pomocí experimentů dále upravovat tak, až bude model dostatečně podobný realitě. Poté se bude experimentovat pomocí změn těchto hodnot pro zjištění optimalizací systému.

Výsledkem experimentu bude histogram zobrazující dobu, kterou transakce stráví v systému. To znamená dobu do doručení transakce přímo adresátovi. Následná doba vyřizování příslušné transakce/dokumentu je v experimentech zanedbána, protože už se nejedná a studavanou problematiku. Z histogramu zjistíme, jaké jsou výsledky. Proč jsou výsledky takové, zjistíme z grafu závislostí vytíženosti jednotlivých pracovníků - kde se hromadí dokumenty, kdo je kdy zpracovává, atd. 

%
%
% POPIS GRAFU TADY ????
%
%

Experimenty budou modelovat průběh padesáti pracovních dnů.

% 5.2 Dokumentace jednotlivých experimentů
\subsection{Dokumentace jednotlivých experimentů}
\subsubsection{Experiment 1}
V tomto experimentu jsme parametry modelu nijak nemodifikovali, očekávali jsme, že výsledky budou odpovídat skutečnosti.
Oproti očekávání se většina transakcí dokončila velice rychle.
Histogram doby zpracování dokumentu v systému je na obrázku \ref{exp1_histogram}.
V~průběhu padesáti pracovních dnů systémem prošlo 2721 transakcí.
Ma\-xi\-mál\-ní doba trans\-ak\-ce strávené v systému byla 19.6 pracovních hodin.
Průměrná doba vyřízení transakce byla 8.01 pracovních hodin. Standardní odchylka 3.87 hodiny.

Z časového průběhu délek front jednotlivých zaměstnanců jsme zjistili nedostatek mo\-de\-lu.
Příchozí dokumenty se zdržely zejména u ředitele, na jednotlivích úsecích firmy se již takřka vůbec nezdržely a téměř okamžitě putovaly ke svým adresátům.
To bylo způsobeno tím, že náměstci byli celou dobu k dispozici k práci na administrativě a nevznikalo u nich zpoždění ve frontách.
Ve skutečnosti je ale administrativa, podobně jako u ředitele, jen část náplně jejich pracovní doby.

V následujících experimentech budeme pracovat s upraveným modelem, ve kterém se tento nedostatek neprojeví.
Oba náměstci se po většinu své pracovní doby budou věnovat jiným pracovním povinnostem a administrativě  
pouze hodinu během jednoho pracovního dne jako ředitel.

\begin{figure}[ht]
 \begin{center}
    \scalebox{0.4}{ \includegraphics{../experimenty/ex1-histogram.png} }
    \caption{1. experiment - histogram doby zpracování dokumentu}
    \label{exp1_histogram}
 \end{center}
\end{figure}


\subsubsection{Experiment 2}
Parametry opět zůstaly nemodifikované, pouze experimentujeme s novou verzí mo\-de\-lu. V této verzi modelu se sekretářka věnuje administrativní činnosti po celou svou pracovní dobu. Ředitel se administrativě věnuje jednu hodinu denně a stejně tak i oba náměstkové. 

Doby transakcí trávících v systému se oproti experimentu č. 1 prodloužili a jsou již velice blízké realitě. Histogram doby zpracování dokumentu v systému z tohoto experimentu je na obrázku \ref{exp2_histogram}.
V tomto experimentu systémem prošlo 2677 transakcí. Maximální doba zpracování transakce byla 36.2 pracovních hodin. Průměrná doba zpracování transakce byla 15.9 hodin. Standardní odchylka 5.6 hodin.

V histogramu vidíme, že první větší výskyt dokončených transakcí je kolem 5. pracovní hodiny. Toto je případ, kdy se všichni hlavní činitelé v systému (sekretářka, ředitel, náměstci) synchronyzují, všichni jsou v práci a věnují se administrativě. Většina transakcí je dokončena mezi 11. a 22. pracovní hodinou, což odpovídá reálnému optimálnímu dokončení transakce.

V dalším experimentu již prověříme model při výraznějších změnách parametrů.

\begin{figure}[ht]
 \begin{center}
    \scalebox{0.4}{ \includegraphics{../experimenty/ex2-histogram.png} }
    \caption{2. experiment }
    \label{exp2_histogram}
 \end{center}
\end{figure}

\subsubsection{Experiment 3}

V tomto experimentu jsme sledovali reakci systému na mimořádně dlouhou poruchu zařízení, se kterým pracuje sekretářka.
Modelovali jsme poruchu zařízení, kterou se podaří odstranit až po 3 dnech.
Cílem experimentu je zjištění, za jak dlouho se systém po takové poruše vrátí do normálního provozu.

Graf závislosti délky front na čase ja znázorněn na obrázku \ref{exp3_graf}. Červenou a zelenou barvou jsou zobrazeny
počty dokumentů, které mají ředitel a sekretářka ve frontě k vyřízení. Modrá barva pro názornost zobrazuje jednotlivé
dny (každá půlperioda ob\-dél\-ní\-ko\-vé\-ho signálu je jeden den). Růžovou barvou je znázorněno trvání poruchy.

Z grafu je patrné, že po odstranění poruchy se podaří sekretářce vyrovnat s takovou mimořádnou situací až po 13 dnech. Řediteli by to trvalo ještě déle, to je způsobeno také tím, že ne každý den se administrativě může věnovat.

Ze statistik dále vyplynulo, že průměrná doba, kterou dokument čeká na zpracování sekretářkou se z 1.5 hodin při
normálním nastavení modelu zvýšil na 6.7 hodin.

Takto mimořádná událost by ale určitě byla individuálně řešena vedením společnosti, aby následky takového výpadku
byly co nejmenší. Mohlo by se jednat třeba o zajištění náhradního zařízení, aby výpadek netrval tak dlouhou dobu, případně
dočasná výpomoc sekretářce dalším pracovníkem. Tento scénář je pouze hypotetický.

\begin{figure}[ht]
 \begin{center}
    \scalebox{0.6}{ \includegraphics{../experimenty/ex3-graf.png} }
    \caption{3. experiment - časový průběh délek front}
    \label{exp3_graf}
 \end{center}
\end{figure}


% 5.3 Závěr experimentů
% Co bylo experimentováním zjištěno
% Jaké chyby v modelu byly odstraněny (oproti původním předpokladům ... došlo ke změně koncepce ... protože ..
\subsection{Závěr experimentů}
Oproti původním předpokladům došlo ke změně koncepce, protože doba setrvání transakcí v systému byla příliš nízká a neodpovídala realitě. Bylo to způsobeno přílišný zjednodušením reality a zanedbáním faktu, že se pracovníci věnují i jiné práci než jen administrativě (kromě sekretářky). Změna se týkala hlavně pracovní činnosti náměstků.

Byly provedeny 3 hlavní experimenty. První experiment byl testovací. Druhý experiment odpovídá reálnému provozu firmy Vodovody a kanalizace Vyškov. Třetí experiment ukazuje, co se stane při dlouhodobější poruše technického zařízení, jaké to bude mít důsledky do budoucna. Experimenty bylo ukázáno, jak se zvýši pracovní zatížení zaměstanců společnosti v závislosti na intenzitě příchodů dokumentů k vyřízení. Z experimentů lze odvodit, že chování systému je dostatečně věrohodné.

Dalšími experimenty by bylo možné zjistit dlouhodobé reakce systému na jakékoliv vnější i vnitřní nezvyklé podněty, jako například v experimentu č.3. 


%%%%%%%%%%%%%%%%%%%%%%%%%%%%%%%%%%%%%%%%%%%%%%%%%%%%%%%%%%%%%%%%%%%%%%%%%%%%%%%%%%%%%%%%=
\newpage


% Kapitola 6 - 
\section{Shrnutí simulačních experimentů a závěr}

Vytvořili jsme model administrativní činnosti pracovníků společnosti Vodovody a kanalizace Vyškov.
Studií provedenou na našem modelu bylo jednoznačně prokázáno, že doba mezi přijetím dokumentu do firmy a doručením přímému adresátovi je velice závislá jak na všech zaměstnancích věnujících se administrativě, tak také na funkčnosti všech zařízeních potřebných pro administrativní práci. Bylo prokázáno, že při jakémkoliv přerušení nebo změně běžného pracovního řádu se tato změna projeví ve dlouhodobém měřítku, kdy zaměstnanci musí dohánět nedodělanou práci.

Z experimentů vyplývá jednoznačné doporučení, aby se provozovatel snažil udržet funkci systému bez zbytečných chyb, které by se pak dlouhodobě projevili. Zlepšení by jednoznačně přineslo zvýšení doby, po kterou se zaměstnanci věnují administrativní činnosti. Poté by nevznikaly takové prodlevy přidoručování dokumentů.


% bla bla ke kazdemu modelu nejaka veticka a bude to hotove

% Jednoznačná odpověď na prvotní otázku studie.
%  Studií provedenou na našem modelu bylo jednoznačně prokázáno/vyvráceno, že ...
%  V rámci experimentů bylo zjištěno, že průměrné zatížení ... je ...
%  Z experimentů vyplývá jednoznačné doporučení, aby provozovatel ... rozšířil výrobu o ...
%  Ze statisticky zpracovaného měření v terénu plyne, že proces příchodů ... se řídí normálním rozložením se středem a ....
% • Na přiložených demo-příkladech jsme ověřili funkčnost ...

%%%%%%%%%%%%%%%%%%%%%%%%%%%%%%%%%%%%%%%%%%%%%%%%%%%%%%%%%%%%%%%%%%%%%%%%%%%%%%%%%%%%%%%%=
\newpage

% Reference
%\renewcommand{\refname}{Literatura}
%\cite{voj,has}
\begin{thebibliography}{99}

\bibitem{ims-sho} P. Peringer.
\textit{Modelování a simulace: Systémy hromadné obsluhy.}
Dostupné z: \textless https://www.fit.vutbr.cz/study/courses/IMS/public/prednasky/IMS.pdf\textgreater.
Strana 139.

\bibitem{ims-zakladni-pojmy} P. Peringer.
\textit{Modelování a simulace: Zakladni pojmy.}
Dostupné z: \textless https://www.fit.vutbr.cz/study/courses/IMS/public/prednasky/IMS.pdf\textgreater.
Strana 7-8.

\bibitem{ims-chovani-systemu} P. Peringer.
\textit{Modelování a simulace: Chování systému.}
Dostupné z: \textless https://www.fit.vutbr.cz/study/courses/IMS/public/prednasky/IMS.pdf\textgreater.
Strana 24.

\bibitem{ims-rozlozeni} P. Peringer.
\textit{Modelování a simulace: Nekterá často používaná rozložení.}
Dostupné z: \textless https://www.fit.vutbr.cz/study/courses/IMS/public/prednasky/IMS.pdf\textgreater.
Strana 90-99.

\bibitem{ims-zarizeni} P. Peringer.
\textit{Modelování a simulace: ??? Zařízení s vylučným přístupem ???.}
Dostupné z: \textless https://www.fit.vutbr.cz/study/courses/IMS/public/prednasky/IMS.pdf\textgreater.
Strana ???.

\bibitem{simlib} P. Peringer, D. Leska, D. Martinek.
\textit{SIMLIB/C++  ==  SIMulation LIBrary for C++.}
Dostupné z: \textless http://www.fit.vutbr.cz/~peringer/SIMLIB/\textgreater.


\end{thebibliography}

\end{document}
