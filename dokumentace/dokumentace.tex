%%
% Name: IMS - Vodovody a kanalizace Vyškov
% Autors: Michal Cupak, Ales Dujicek
% Date: 2012-12-07
%%

%\documentclass[11pt,a4paper]{article}
\documentclass[a4paper]{article}
\usepackage[a4paper]{geometry}
\usepackage[czech]{babel}
%\usepackage{czech}
%\usepackage[latin2]{inputenc}
%\usepackage[IL2]{fontenc}
\usepackage[utf8]{inputenc}
%\usepackage[T1]{fontenc}
\usepackage{times}

\usepackage{graphics}
\usepackage{picture}
%\usepackage[czech,ruled]{algorithm2e}

\usepackage{amsmath}
\usepackage{amsthm}
\usepackage{amsfonts}

%\newcommand{\uvoz}[1]{\quotedblbase #1\textquotedblleft}

\begin{document}

% Úvodní strana
%%
% Name: Titulní strana
% Autor: Michal Cupak
% Date: 2011-3-11
%%


\begin{titlepage}

\begin{center}
  
		{\Huge  {\scshape Vysoké učení technické v Brně}\\}
		{\huge {\scshape Fakulta informačních technologií}\\}
	
	\vspace{\stretch{0.382}}
		
	{\LARGE Modelování a simulace}
	\\
	\medskip
	{\Huge Vodovody a kanalizace Vyškov}\\
	\vspace{\stretch{0.618}}
\end{center}
{\Large \today \\\\ Michal Cupák \\ Aleš Dujíček}


\end{titlepage}


% Obsah
\thispagestyle{empty}
\tableofcontents
\newpage

% Kapitola 1 - Úvod
\section{Úvod}
V této práci je řešena implementace systému hromadné obsluhy, která bude použita pro 
sestavení modelu zpracování dokumentů v administrativě firmy Vodovody a Kanalizace Vyškov.\\
\\
Na základě modelu a simulačních experimentů bude ukázáno 
chování systému toku dokumentů mezi jednotlivými zaměstnanci firmy v podmínkách běžné denní činnosti firmy.\\ 
\\
Smyslem experimentů je demonstrovat, jak se při změně jakékoliv veličiny ovlivňující tok dokumentů ve studované firmě změní délka zpracování daného dokumentu či vytížení ostatních zaměstnanců studované firmy.\\
\\
Správnost zvolené koncepce byla ověřena přímo se zaměstnancem studované firmy.\\
\\
\textit{(pro zpracování modelu bylo nutno nastudovat ..., zpracovat, ... model je ve svém oboru zajímavý/ojedinělý v ...)}

\subsection{Kdo se na práci podílel}
Autory této práce jsou studenti Fakulty Informačních Technologií Vysokého Učení Technického v Brně. Jmenovitě to jsou pánové Michal Cupák a Aleš Dujíček.\\
\\
Dodavatelem odborných faktů byl ing. Oldřich Novoměstský, Vedoucí technického úseku ve studované firmě Vodovody a Kanalizace Vyškov.\\
\\
Odborným konzultantem byl taktéž ing. Novoměstský ve spolupráci s paní Annou Sýkorovou, sekretářkou ředitele.

\subsection{Ověření validity modelu}
Ověřování validity modelu probíhalo při osobních setkáních s ing. Novoměstským. Jednalo se především o vysvětlení problematiky zpracování dokumentů ve studované firmě. Dále pak o ověření funkčnosti modelu pomocí konzultace grafů vzniklých ze simulace.\\
\\
Počet a procentuální rozdělení příchozích dokumentů byl konzultován s paní Sýkorovou, která má tyto dokumenty zpočátku na starosti.

\newpage


% Kapitola 2 - 
\section{Rozbor tématu a použitých metod/technologií}
\newpage


% Kapitola 3 - 
\section{Koncepce - modelářská témata}
\newpage


% Kapitola 4 - 
\section{Architektura simulačního modelu/simulátoru}
\newpage


% Kapitola 5 - 
\section{Podstata simulačních experimentů a jejich průběh}
\newpage


% Kapitola 6 - 
\section{Shrnutí simulačních experimentů a závěr}
\newpage


\end{document}

